\documentclass[margin,line]{res}
\usepackage{hyperref}
\hypersetup{
    colorlinks=true,
    linkcolor=blue,
    filecolor=magenta,      
    urlcolor=cyan,
    }
\usepackage{enumitem}
\usepackage{url}
\oddsidemargin -.5in
\evensidemargin -.5in
\textwidth=6.0in
\itemsep=0in
\parsep=0in
\topmargin=0in
\topskip=0in
\newenvironment{list1}{
  \begin{list}{\ding{113}}{%
      \setlength{\itemsep}{0in}
      \setlength{\parsep}{0in} \setlength{\parskip}{0in}
      \setlength{\topsep}{0in} \setlength{\partopsep}{0in}
      \setlength{\leftmargin}{0.17in}}}{\end{list}}
\newenvironment{list2}{
  \begin{list}{$\bullet$}{%
      \setlength{\itemsep}{0in}
      \setlength{\parsep}{0in} \setlength{\parskip}{0in}
      \setlength{\topsep}{0in} \setlength{\partopsep}{0in}
      \setlength{\leftmargin}{0.2in}}}{\end{list}}


    
\begin{document}

\centerline{\Large CURRICULUM VITAE}

\name{\LARGE \sc{Uddeepta Deka}} %\hfill {\em \today}

\begin{resume}
\section{\sc Contact Information}

\vspace{.05in}
\begin{tabular}{@{}p{3.5in}p{3in}}
International Centre for Theoretical Sciences             & {Phone:}  (91) 80 4653 6162 \\
Tata Institute of Fundamental Research
 & {E-mail:}  uddeepta.deka@icts.res.in\\
Survey No. 151, Shivakote \\
Hesaraghatta Hobli
Bengaluru, INDIA, 560089
\end{tabular}


\section{\sc Research Interests}

Gravitational Wave Astrophysics, Gravitational Lensing, Cosmology, Scientific Computing

\section{\sc Education}
{\bf International Centre for Theoretical Sciences, Tata Institute of Fundamental Research}, Bengaluru, Karnataka. India \hfill Aug 2019 -- present\\
Ph.D. Research Fellow

{\bf University of Delhi}, North Delhi, Delhi. India \hfill Aug 2018 -- July 2019\\
CSIR-JRF Fellow

{\bf University of Delhi}, North Delhi, Delhi. India \hfill Aug 2016 -- July 2018\\
%\vspace*{-.1in}
M.Sc. Physics 

{\bf Ramjas College, University of Delhi}, North Delhi, Delhi. India \hfill July 2013 -- July 2016\\
B.Sc. (honors) Physics 

%{\bf Shrimanta Shankar Academy, CBSE\footnote{Central Board of Secondary Education}}, Dispur, Assam. India \hfill July 2011 -- Apr 2013\\
%\vspace*{-.1in}
%Senior Secondary - 12th standard 

%{\bf Lotus Academy, CBSE}, Doomdooma, Asssam. India \hfill Apr 1998 -- Apr 2011\\
%High School - 10th standard 

\section{\sc Pre-Ph.D. Research Experience}
{\bf University of Delhi}, North Delhi, Delhi, India. \hfill{Jan 2018 -- Jun 2018}\\
{\ Masters Dissertation}\\
Mentor - {\em Prof. Patrick Das Gupta}\\
Project - {\em Statistical Studies of Pulsar Transverse Velocities and the Nulling Phenomena}\\

{\bf NCRA-TIFR\footnote{National Centre for Radio Astrophysics - Tata Institute of Fundamental Research}}, Pune, Maharashtra. India. \hfill{May 2017 -- July 2017}\\
{\em Summer Research Fellow}, Indian Academies of Sciences\\
Mentor - {\em Dr. Sushan Konar}\\
Project - {\em Statistical Studies of Nulling Radio Pulsars}\\

{\bf University of Delhi}, North Delhi, Delhi. India. \hfill{Aug 2015 -- Oct 2016}\\
{\em Student Researcher}, Innovation Project 2015-16\\
Mentor - {\em Dr. N. Rathnasree}, Director, Nehru Planetarium, New Delhi\\
Project Supervisor - {\em Dr. Ashok Kumar}, Associate Professor, Ramjas College, Univ. of Delhi\\
Project - {\em An Interdisciplinary Study of Light Pollution in Indian Context}\\

{\bf SINP\footnote{Saha Institute of Nuclear Physics}}, Kolkata, West Bengal. India. \hfill{May 2015 -- Jun 2015}\\
{\em Summer Associate}, Undergraduate Associateship programme\\
Mentor - {\em Prof. Pradip K. Roy}, HENPP\footnote{High Energy Nuclear and Particle Physics} Division\\
Project - {\em Understanding the basics of Quantum Field Theory}\\

{\bf SINP}, Kolkata, West Bengal. India. \hfill{May 2014 -- Jul 2014}\\
{\em Summer Associate}, Undergraduate Associateship programme\\
Mentor - {\em Prof. Parthasarathi Mitra}, Theory division\\
Project - {\em Solving mathematical problems analytically and using numerical techniques}\\

\section{\sc Publications}
\begin{itemize}[noitemsep]
{
\item Konar, S., Deka, U.,\href{https://www.ias.ac.in/article/fulltext/joaa/040/05/0042}{``Radio pulsar sub-populations (I): The curious case of nulling pulsars"}, \textit{Journal of Astrophysics and Astronomy, Indian Academy of Sciences}, 2019.
}\end{itemize}

\section{\sc Awards and Fellowships}

{\bf Qualified GATE 2019\footnote{Graduate Aptitude Test in Engineering}}: National level test conducted by the IITs, for pursuing postgraduate degree / Ph.D. in engineering and natural sciences. \hfill{Apr 2019} \\

{\bf CSIR-JRF\footnote{Council of Scientific and Industrial Research - Junior Research Fellowship}}: National level test jointly conducted by CSIR and UGC, India, for pursuing PhD. in India and carries a scholarship and research grant. \hfill{Apr 2018} \\

{\bf Delhi University Faculty of Science Meritorious Award 2016-17}: securing position among Top 10 students in M.Sc. Physics class in semesters 1\& 2. \hfill{Feb 2018} \\

{\bf UGC-JRF\footnote{University Grants Commission - Junior Research Fellowship}}: National level test jointly conducted by CSIR and UGC, India, for pursuing PhD. in India and carries a scholarship and research grant. \hfill{Apr 2017} \\

{\bf Delhi University Faculty of Science Meritorious Award 2013-16}: securing position among Top 10 students in B.Sc. Physics. \hfill{Feb 2017} \\

{\bf Ramjas college Academic Award - Gold Medal}: securing highest score in B.Sc.(honors) Physics, Ramjas College. \hfill{Jan 2017}

\section{\sc Other Achievements}
\begin{itemize}[noitemsep]
{
%\item Secured First Class 5th position in M.Sc. Physics, 2016-18, Department of Physics and Astrophysics, University of Delhi.
\item Represented Ramjas College in Quest 2015, organized by the Centre for Science Education and Communication, University of Delhi.
\item Participated in JENESYS 2.0 (Science and Technology 23rd Batch), conducted by Japan International Cooperation Centre, in January 2015.
\item Secured All India 6th position, 49th Annual All-India UN Information Test 2006, conducted by the United Schools Organization of India, VEC for the United Nations.
}\end{itemize}

\section{\sc Teaching Experience}
\begin{itemize}[noitemsep]
\item Teaching Assistant to Prof. Thomas W. Baumgarte, \textit{ICTS Summer School on Gravitational Wave Astronomy}\hfill{Jul-Aug 2023}
\item Mentor for the \href{https://www.icts.res.in/discussion-meeting/gwodw2023}{Gravitational Wave Open Data Workshop study hub} at ICTS, Bengaluru.\\ \hfill{May 2023}
\item Teaching Assistant to Prof. Bernhard Mueller, \textit{ICTS Summer School on Gravitational Wave Astronomy}\hfill{May-Jun 2022}
\item Teaching Assistant to Prof. Vuc Mandic, \textit{ICTS Summer School on Gravitational Wave Astronomy}\hfill{Jul 2021}
\item Teaching Assistant to Prof. Loganayagam R., \textit{Advanced Electrodynamics} \hfill{Feb-Jul 2021}
\item Teaching Assistant to Prof. R. Nityananda, \textit{ICTS Summer course - Light and Beyond} \hfill{Jul-Aug 2020} 
\end{itemize}

\section{\sc Workshops and Research Schools Attended}
\begin{itemize}[noitemsep]
{
\item \href{https://www.icts.res.in/program/gws2023}{Summer School on GW Astronomy}. ICTS, Bengaluru. \hfill{Jul-Aug 2023}
\item (School) \href{https://icts.res.in/program/BigDataCosmo}{Largest Cosmological Surveys and Big Data Science}. ICTS, Bengaluru.\hfill{May 2023}
\item (Discussion Meeting) \href{https://www.icts.res.in/discussion-meeting/LGWD}{Lunar Gravitational-Wave detection}. ICTS, Bengaluru. \hfill{Apr 2023}
\item \href{https://www.icts.res.in/program/gws2022}{Summer School on GW Astronomy}. ICTS, Bengaluru. \hfill{May-Jun 2022}
\item (School) \href{https://www.icts.res.in/program/peu2022}{Physics of the Early Universe}. ICTS, Bengaluru. \hfill{Jan 2022} 
\item \href{https://www.icts.res.in/program/gws2021}{Summer School on GW Astronomy}. ICTS, Bengaluru. \hfill{Jul 2021}
\item (Workshop) \href{https://events.iitgn.ac.in/2020/TGRGW/}{Testing General Relativity using Gravitational Waves}. IIT, Gandhinagar. \hfill{Aug 2020}
\item \href{https://www.gw-openscience.org/static/workshop3/}{GW Open Data workshop 3}. LIGO-VIRGO collaboration. \hfill{May 2020}
\item \href{https://www.icts.res.in/program/gws2020}{Summer School on GW Astronomy}. ICTS, Bengaluru. \hfill{May 2020}
\item (Discussion Meeting) \href{https://www.icts.res.in/discussion-meeting/smbh2019}{Astrophysics of Supermassive Black Holes}. ICTS, Bengaluru. \hfill{Dec 2019}
\item \href{https://www-apps.iucaa.in/Stu-Prog-ISSIA.html}{IUCAA Summer School for Astronomy and Astrophysics}. IUCAA, Pune. \hfill{May -- Jun 2016}
\item (Seminar) Hundred Years of Einstein's Theory of Relativity, University of Delhi, Delhi. \hfill{2015}
}\end{itemize}


\section{\sc Talks in Conferences}
\begin{itemize}[noitemsep]
{
\item \textit{Micro-lensed gravitational wave signals by compact objects: effect of the galaxy lens}, \href{https://www.iiserkol.ac.in/~iagrg32/}{32nd meeting of the Indian Association for General Relativity and Gravitation. IISER, Kolkata}. \hfill{Dec 2022}
}\end{itemize}

\section{\sc Invited talks}
\begin{itemize}[noitemsep]
{
\item \href{https://github.com/singhmukesh1729/EPO_DTU2023}{\textit{Workshop on Gravitational Wave data analysis\footnote{As part of LIGO-India Education and Public Outreach}}}, Delhi Technological University, Delhi.\hfill{Feb 2023}
}\end{itemize}

\section{\sc Technical Skills}
Proficient in : Python, C++, C, Mathematica\\
%{\bf Operating Systems}: Linux, Windows\\

\section{\sc Extra-Curricular Activities}
\begin{itemize}[noitemsep]
{
\item Worked as `Science Teacher and Communicator' at Bravo, an NGO which caters to the educational needs of underprivileged children.
\item General Secretary (2014-15), Physics Society, Ramjas College. (Organized seminars, talks, physics lecture series, paper presentations, debates, quizzes etc.).
\item Member, Equal Opportunity Cell, Ramjas College (Objective: Service to Differently Abled students).
\item Member, National Service Scheme, Ramjas College Chapter (Objective: Social service).
\item Painting (qualified III year –- Ankan Ratna, North East Art Academy, under Society of Art and Craft, Kolkata).
\item Music Instrumental(Percussion) –- Tabla (Visharad, Bhatkhande University, Lucknow).
\item Sports: Badminton and chess player.
}
\end{itemize}

\section{\sc Hobbies and Interests}
Reading novels, playing percussion intruments or flute and painting.\\ Writing poems and articles. \href{http://megh-ud.blogspot.com/}{Personal Blog}

\end{resume}
\end{document}